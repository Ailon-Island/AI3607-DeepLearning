% !Mode:: "TeX:UTF-8"
% !TEX program  = xelatex
\documentclass[a4paper]{article}
\usepackage{ctex}
\usepackage[left=1.5cm, right=1.5cm, top=1.5cm, bottom=1.5cm]{geometry} %页边距
\usepackage{helvet}
\usepackage{amsmath, amsfonts, amssymb} % 数学公式、符号
\usepackage[english]{babel}
\usepackage{graphicx}   % 图片
\usepackage{url}        % 超链接
\usepackage{bm}         % 加粗方程字体
\usepackage{multirow}
\usepackage{booktabs}
\usepackage{tikz}%调用宏包tikz
\usepackage{circuitikz}%调用宏包circuitikz
\usepackage{enumerate}
\usepackage{algorithm}
\usepackage{algorithmicx}
\usepackage{algpseudocode}
\usepackage{graphicx}
\usepackage[hidelinks]{hyperref}
\usepackage{listings}
\usepackage{textcomp}
\usepackage{multicol}
\usepackage[backend=biber,style=numeric,sorting=none]{biblatex}
\addbibresource{reference.bib}

% Python listing
\newcommand\pythonstyle{\lstset{
language=Python,
basicstyle=\sffamily,
keywordstyle=\textbf,
commentstyle=\color{blue},
showstringspaces=false, 
numbers=left }}
% Python environment
\lstnewenvironment{python}[1][]{
\pythonstyle \lstset{#1} }{}

\newcommand{\threemetrics}[3]{\multirow{3}{*}{\shortstack[c]{$\textcolor{orange}{#1}$\\$\textcolor{blue}{#2}$\\$\textcolor{green}{#3}$}}}
\newcommand{\twometrics}[2]{\multirow{2}{*}{\shortstack[c]{$\textcolor{blue}{#1}$\\$\textcolor{green}{#2}$}}}

\renewcommand{\algorithmicrequire}{ \textbf{Input:}}       
\renewcommand{\algorithmicensure}{ \textbf{Output:}} 
%算法格式
\usepackage{subfigure}
\usepackage{fancyhdr} %设置页眉、页脚
\usepackage{gensymb}

\pagestyle{fancy}
\lhead{MiniProject 2: Image Classification, AI3607 Deep Learning and Its Application}
\chead{}
\rhead{蒋伟, 520030910149}
\lfoot{}
\cfoot{\thepage}
\rfoot{全部实验结果详见 \texttt{./checkpoint} 目录。}


\usepackage{ifthen}
\usepackage{xifthen}

\newcommand{\dom}[1]{\mathop{\mathrm{dom}}\left(#1\right)}
\newcommand{\rng}[1]{\mathop{\mathrm{rng}}\left(#1\right)}
\newcommand{\preimg}[2][]{ \ifthenelse{\isempty{#1}}
  {\mathop{\mathrm{preimage}}\left(#2\right)}
  {\mathop{\mathrm{preimage}}_{#1}\left(#2\right)} }
\newcommand{\set}[1]{\left\{#1\right\}}

\newenvironment{proof}{{\par\noindent\it Proof.}\quad\par}{\hfill $\square$\par}  

\title{AI3607 Deep Learning and Its Application\\MiniProject 2: Image Classification}
\author{\sffamily 蒋伟, F2003801, 520030910149}
\date{(Dated: \today)}
\begin{document}
\section{目标任务}
\begin{enumerate}
    \item 利用神经网络完成 CIFAR-10 数据集的图像分类任务。
    \item 将训练集划分,所有类别小于等于 $5$ 的数据仅保留 10\%,剩余部分不变。重新训练,比较结果,并尝试改进。
\end{enumerate}

\section{CIFAR-10 数据集}
数据集包含了 10 类不同物体的图片($32\times 32$, RGB)和对应的类别标签。
训练集中每类数据各 5,000 对,共 50,000 对数据;测试集中每类数据各 1,000 对,共计 10,000 对数据。

为了对输入图片进行分类,模型的输入是 $32\times 32$ 大小的 RGB 图片,输出则为类别预测。

\section{CIFAR-10 数据集图像分类}
\subsection{网络结构}
\subsubsection{ResNet50}
首先,尝试了 ResNet50 \cite{resnet} 作为分类网络,效果优异,但是训练较慢,为更好地进行对比,故使用更简单的网络。
\subsubsection{Simple Classifier}
我设计了更简单的基于卷积的分类网络,其网络结构如下。它也能达到较好的效果。

\begin{multicols}{2}
    \begin{lstlisting}[language=Python, basicstyle=\ttfamily, numbers=left]
nn.Conv2d(3, 64, 5)
nn.BatchNorm2d(64)
nn.ReLU()
nn.MaxPool2d(2, 2)
nn.Conv2d(64, 256, 5)
nn.BatchNorm2d(256)
nn.ReLU()
nn.MaxPool2d(2, 2)
nn.Linear(6400, 256)
nn.ReLU()
nn.Linear(256, 96)
nn.ReLU()
nn.Linear(96, 10)
    \end{lstlisting}
\end{multicols}

\subsection{对抗过拟合}
训练过程中出现明显过拟合(训练损失下降,测试损失不变),故尝试多种方法解决。其中包括数据增强(随机翻转,随机裁剪)和 weight decay。前者从数据层面,后者从参数层面,共同引导模型关注关键特征,提升泛化能力。

\section{非平衡数据集(CIFAR-10 IMBAlanced)图像分类}
数据划分后,不同类别间出现不平衡。且训练集和测试集分布偏移。直接训练会导致不同类的关注度差异较大,直接表现为放大的劣势类错误量带来的高测试错误率。
\subsection{再平衡数据集(CIFAR-10 REBAlanced)}
原计划使用带权交叉熵损失,对不同类的惩罚加权,从而平衡各类对模型的影响。但 \texttt{Jittor} 的带权交叉熵行为不稳定(实测全一权重和不输入权重的训练结果不同),此外也避免劣势类异常点导致梯度爆炸,就改为对劣势类过采样,使得训练集中各类别的数据量相同的方法。由此得到的数据集称为再平衡数据集。

\subsection{重述:对抗过拟合}
REBA 数据集中存在大量重复样本,因此极其容易过拟合。此时一方面采用数据增强增加样本多样性,另一方面采用 weight decay 限制参数的增长,在均等关注各类的同时保障模型的泛化能力。
\section{实验结果}
实验结果如表 \ref{tab:1}, \ref{tab:2}, \ref{tab:3} 所示,此外,在表 \ref{tab:3} 的设置下最佳的实验结果为 F.3+C 和 weight decay 为 $5e-3$ 的组合,训练损失 $0.5177$, 测试损失 $0.8331$,测试准确率 $\mathbf{72.73\%}$,比 IMBA 数据集的直接训练结果提升$10.64\%$。
\begin{multicols}{2}
    \begin{table}[H]
        \centering
        \begin{tabular}{ccc}
            \hline
            Model&ResNet50&Simple\\
            \hline
            Loss& $\mathbf{0.5786}$&$0.6540$\\
            Acc.&$\mathbf{86.00\%}$&$78.20\%$\\
            \hline
            
        \end{tabular}
        \caption{Result of different models on CIFAR-10. Test losses and accuracies are reported.}
        \label{tab:1}
    \end{table}

    \begin{table}[H]
        \centering
        \begin{tabular}{cccc}
            \hline
            Variation&Original&IMBA&REBA\\
            \hline
            Loss&$0.6540$&$1.4708$&$1.7856$\\
            Acc.&$78.20\%$&$62.09\%$&$62.68\%$\\
            \hline
        \end{tabular}
        \caption{Result on variations of CIFAR-10 with Simple Classifier.}
        \label{tab:2}
    \end{table}

\end{multicols}

\begin{table}[H]
    \centering
    \begin{tabular}{ccccccc}
        \cline{3-7}
        \multicolumn{2}{c}{\threemetrics{Train\ Loss}{Test\ Loss}{Test\ Acc.}}&\multicolumn{5}{c}{ Augmentation}\\
        &&None&C&F.3+C&F.5+C&F.9+C\\
        \cline{3-7}
        \multicolumn{1}{|c}{\multirow{9}{*}{\rotatebox{90}{Weight Decay}}}
        &\multicolumn{1}{c|}{\multirow{2}{*}{$0$}}&
        \threemetrics{0.1198}{2.7436}{57.57\%}&\threemetrics{0.5312}{1.3350}{60.60\%}&\threemetrics{0.7723}{1.1128}{63.38\%}&\threemetrics{0.8797}{1.1291}{61.61\%}&\multicolumn{1}{c|}{\threemetrics{0.8810}{1.3040}{55.70\%}}\\
        \multicolumn{1}{|c}{}&\multicolumn{1}{c|}{}&&&&&\multicolumn{1}{c|}{}\\
        \multicolumn{1}{|c}{}&\multicolumn{1}{c|}{}&&&&&\multicolumn{1}{c|}{}\\
        \multicolumn{1}{|c}{}
        &\multicolumn{1}{c|}{\multirow{3}{*}{$1e-3$}}&
        \threemetrics{0.0263}{1.7856}{62.68\%}&\threemetrics{0.1329}{1.3967}{66.66\%}&\threemetrics{0.3496}{1.0275}{69.99\%}&\threemetrics{0.6150}{0.9844}{66.67\%}&\multicolumn{1}{c|}{\threemetrics{0.4338}{1.1705}{64.12\%}}\\
        \multicolumn{1}{|c}{}&\multicolumn{1}{c|}{}&&&&&\multicolumn{1}{c|}{}\\
        \multicolumn{1}{|c}{}&\multicolumn{1}{c|}{}&&&&&\multicolumn{1}{c|}{}\\
        \multicolumn{1}{|c}{}
        &\multicolumn{1}{c|}{\multirow{3}{*}{$1e-2$}}&
        \threemetrics{0.8659}{1.1954}{64.73\%}&\threemetrics{0.4323}{0.9248}{69.04\%}&\threemetrics{0.6833}{\mathbf{0.8616}}{\mathbf{70.40\%}}&\threemetrics{0.8101}{0.9221}{67.84\%}&\multicolumn{1}{c|}{\threemetrics{0.8659}{1.1295}{59.05\%}}
        \\
        \multicolumn{1}{|c}{}&\multicolumn{1}{c|}{}&&&&&\multicolumn{1}{c|}{}\\
        \multicolumn{1}{|c}{}&\multicolumn{1}{c|}{}&&&&&\multicolumn{1}{c|}{}\\
        \cline{3-7}
    \end{tabular}
    \caption{Result of anti-overfitting methods with Simple Classifier on CIFAR-10 REBA. F.X stands for random flip (vertical and horizontal) with total probability of $0.X$. C stands for random crop.}
    \label{tab:3}
\end{table}

\section{附录}
\subsection{训练细节}
默认情况下:学习率 $1e-2$,batch size $1024$,共训练 $5,000,000$ 迭代,从第 $50,000$ 次迭代起,每 $200,000$ 迭代降低学习率为 $0.5$ 倍。weight decay $1e-3$。数据增强不启用;若启用,方式为随机翻转和随机裁剪。

\subsection{数据增强的使用场景}
数据增强能增加样本多样性从而提高表现,但它是基于数据的,因此受到数据集本身影响。在 IMBA 数据集上,类别不均导致增强的表现集中体现在优势类,损失下降的同时准确性也下降了。

数据增强对模型的表达能力有要求。在 CIFAR-10 数据集上,数据增强使 Simple Classifier 的性能不升反减,却能使 ResNet50 的性能提升(Loss: $0.3838$, Acc.: $87.35\%$)。

\subsection{IMBA 数据集的实现}
直接使用 mask 会导致 batch size,引发训练不稳定,同时也不能保证均匀、准确地删去九成劣势数据。同时为了便于实现 REBA 数据集,我采用先删减后读入的方式实现 IMBA 数据集。

\end{document}